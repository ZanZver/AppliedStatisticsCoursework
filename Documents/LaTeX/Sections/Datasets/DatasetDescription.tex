
\subsection{Dataset description}\label{subsection}
describe and justify the datasets used to show what information is contained in the datasets and how it aids to answer the relevant questions. If the description is very long, then summarise the key information and detail the columns in the datasets in appropriate appendices. Also, properly cite the dataset in the References section and include their exact hyperlinks; \newline
(Data Pre-processing): discuss any data pre-processing that is required to prepare the data for subsequent analysis. 

\begin{center}
\begin{longtable}{|m{6em}|m{18em}|m{7em}|}
    \hline
        Column Name & 
        Description & 
        Type\\
    \hline
        ID & 
        Unique identifier for the record. & 
        Number\\
    \hline
        Case Number & 
        The Chicago Police Department RD Number (Records Division Number),
        which is unique to the incident. & 
        Plain Text\\
    \hline
        Date & 
        Date when the incident occurred. this is sometimes a best estimate. & 
        Date and Time\\
    \hline
        Block & 
        The partially redacted address where the incident occurred, 
        placing it on the same block as the actual address. & 
        Plain Text\\
    \hline
        IUCR & 
        The Illinois Unifrom Crime Reporting code. This is directly linked 
        to the Primary Type and Description. See the list of IUCR codes 
        at https://data.cityofchicago.org/d/c7ck-438e. & 
        Plain Text\\
    \hline
        Primary Type & 
        The primary description of the IUCR code. & 
        Plain Text\\
    \hline
        Description & 
        The secondary description of the IUCR code, 
        a subcategory of the primary description. & 
        Plain Text\\
    \hline
        Location Description & 
        Description of the location where the incident occurred. & 
        Plain Text\\
    \hline
        Arrest & 
        Indicates whether an arrest was made. & 
        Checkbox\\
    \hline
        Domestic & 
        Indicates whether the incident was domestic-related as 
        defined by the Illinois Domestic Violence Act. & 
        Checkbox\\
    \hline
        Beat & 
        Indicates the beat where the incident occurred. A beat is the 
        smallest police geographic area – each beat has a dedicated 
        police beat car. Three to five beats make up a police sector, 
        and three sectors make up a police district. The Chicago Police 
        Department has 22 police districts. See the beats 
        at https://data.cityofchicago.org/d/aerh-rz74. & 
        Plain Text\\
    \hline
        District & 
        Indicates the police district where the incident occurred.  
        See the districts at https://data.cityofchicago.org/d/fthy-xz3r. & 
        Plain Text\\
    \hline
        Ward & 
        The ward (City Council district) where the incident occurred. 
        See the wards at https://data.cityofchicago.org/d/sp34-6z76. & 
        Number\\
    \hline
        Community Area & 
        Indicates the community area where the incident occurred. Chicago has 77 community 
        areas. See the community areas at https://data.cityofchicago.org/d/cauq-8yn6. & 
        Plain Text\\
    \hline
        FBI Code & 
        Indicates the crime classification as outlined in the FBI's National Incident-Based 
        Reporting System (NIBRS). See the Chicago Police Department listing of these classifications at
        https://ucr.fbi.gov/nibrs/2011/resources/nibrs-offense-codes/view
        & 
        Plain Text\\
    \hline
        X Coordinate & 
        The x coordinate of the location where the incident occurred in State Plane Illinois 
        East NAD 1983 projection. This location is shifted from the actual location for 
        partial redaction but falls on the same block. & 
        Number\\
    \hline
        Y Coordinate & 
        The y coordinate of the location where the incident occurred in State Plane Illinois 
        East NAD 1983 projection. This location is shifted from the actual location for 
        partial redaction but falls on the same block. & 
        Number\\
    \hline
        Year & 
        Year the incident occurred. & 
        Number\\
    \hline
        Updated On & 
        Date and time the record was last updated. & 
        Date and Time\\
    \hline
        Latitude & 
        The latitude of the location where the incident occurred. This location is shifted 
        from the actual location for partial redaction but falls on the same block. & 
        Number\\
    \hline
        Longitude & 
        The longitude of the location where the incident occurred. This location is shifted 
        from the actual location for partial redaction but falls on the same block. & 
        Number\\
    \hline
        Location & 
        The location where the incident occurred in a format that allows for creation of 
        maps and other geographic operations on this data portal. This location is shifted 
        from the actual location for partial redaction but falls on the same block. & 
        Location\\
    \hline
\caption{Main dataset attributes}
\end{longtable}
\end{center}

Illinois Uniform Crime Reporting (IUCR) \newline
https://data.cityofchicago.org/Public-Safety/Chicago-Police-Department-Illinois-Uniform-Crime-R/c7ck-438e
\begin{center}
\begin{longtable}{|m{6em}|m{18em}|m{7em}|}
    \hline
    Column Name & 
    Description & 
    Type\\
    \hline
    IUCR & 
    Crime ID & 
    Plain Text\\
    \hline
    Primary description & 
    First (main) description of the crime & 
    Plain Text\\
    \hline
    Secondary description & 
    Additional description of the crime & 
    Plain Text\\
    \hline
    Index code & 
    There are 2 codes, "I" (Index) and "N" (Non-Index). "I" code indicates crimes that are collected nation-wide (by FBI) while "N" is for other (usually smaller) crimes. & 
    Plain Text\\
    \hline
    Active & 
    Whether the code is active. Retired codes (No) are present in this dataset for historical reference. There is a filtered view for this dataset showing only active codes. & 
    Boolean\\
    \hline
\caption{IUCR supporting dataset attributes}
\end{longtable}
\end{center}

Beats \newline
https://data.cityofchicago.org/Public-Safety/Boundaries-Police-Beats-current-/aerh-rz74
\begin{center}
\begin{longtable}{|m{6em}|m{18em}|m{7em}|}
    \hline
    Column Name & 
    Description & 
    Type\\
    \hline
    The\_geom & 
    List of locations (as multi polygon) containing LAT and LON. & 
    Multipolygon - location\\
    \hline
    District & 
    ID of the police district & 
    Number\\
    \hline
    Sector & 
    Geographically divided area with associated ID & 
    Number\\
    \hline
    Beat & 
    No description. & 
    Number\\
    \hline
    Beat\_num & 
    Indicates Beat ID from main dataset &
    Number\\
    \hline
\caption{Beats supporting dataset attributes}
\end{longtable}
\end{center}

Police Districts \newline
https://data.cityofchicago.org/Public-Safety/Boundaries-Police-Districts-current-/fthy-xz3r
\begin{center}
\begin{longtable}{|m{6em}|m{18em}|m{7em}|}
    \hline
    Column Name & 
    Description & 
    Type\\
    \hline
    The\_geom & 
    List of locations (as multi polygon) containing LAT and LON where districts operate.  & 
    Multipolygon - location\\
    \hline
    Dist\_label & 
    Name of the district & 
    Plain text\\
    \hline
    Dist\_num & 
    Number of the district & 
    Plain text\\
    \hline
\caption{Police districts supporting dataset attributes}
\end{longtable}
\end{center}

Community Areas \newline
https://data.cityofchicago.org/Facilities-Geographic-Boundaries/Boundaries-Community-Areas-current-/cauq-8yn6
\begin{center}
\begin{longtable}{|m{6em}|m{18em}|m{7em}|}
    \hline
    Column Name & 
    Description & 
    Type\\
    \hline
    The\_geom & 
    List of locations (as multi polygon) containing LAT and LON. & 
    Multipolygon - location\\
    \hline
    Perimeter & 
    No description. & 
    Number\\
    \hline
    Area & 
    No description. & 
    Number\\
    \hline
    Comarea\_ & 
    No description. & 
    Number\\
    \hline
    Comarea\_ID & 
    No description. & 
    Number\\
    \hline
    Area\_numbe & 
    ID of the area. This is the the same as in main dataset. Note, column name does not have r in number. & 
    Number\\
    \hline
    Community & 
    Name of the community / region. & 
    Plain Text\\
    \hline
    Area\_num\_1 & 
    ID of the area. This is the the same as in main dataset. Note, column name does not have r in number. & 
    Number\\
    \hline
    Shape\_area & 
    No description. & 
    Number\\
    \hline
    Shape\_len & 
    No description. & 
    Number\\
    \hline
\caption{Community area supporting dataset attributes}
\end{longtable}
\end{center}


FBI codes \newline
https://ucr.fbi.gov/nibrs/2011/resources/nibrs-offense-codes/view \newline
Group A Offenses
\begin{center}
\begin{longtable}{|m{8em} m{16em} m{7em}|}
    \hline
    Offense Code & 
    Offense Description & 
    Crime Against\\
    \hline
    
    Arson & & \\
    \hline
    200 &
    Arson &
    Property\\
    \hline

    Assault Offenses & & \\
    \hline
    13A &
    Aggravated Assault &
    Person\\
    \hline
    13B &
    Simple Assault &
    Person\\
    \hline
    13C &
    Intimidation &
    Person\\
    \hline

    Bribery & & \\
    \hline
    510 &
    Bribery &
    Property\\
    \hline

    Burglary/Breaking \& Entering & & \\
    \hline
    220 &
    Burglary/Breaking \& Entering &
    Property\\
    \hline

    
    Counterfeiting/Forgery & & \\
    \hline
    250 &
    Counterfeiting/Forgery &
    Property\\
    \hline

    Destruction/Damage/Vandalism of Property & & \\
    \hline
    290 &
    Destruction/Damage/Vandalism of Property &
    Property\\
    \hline

    Drug/Narcotic Offenses & & \\
    \hline
    35A &
    Drug/Narcotic Violations &
    Society\\
    \hline
    35B &
    Drug Equipment Violations &
    Society\\
    \hline

    Embezzlement & & \\
    \hline
    270 &
    Embezzlement &
    Property\\
    \hline

    Extortion/Blackmail & & \\
    \hline
    210 &
    Extortion/Blackmail &
    Property\\
    \hline

    Fraud Offenses & & \\
    \hline
    26A &
    False Pretenses/Swindle/Confidence Game &
    Property\\
    \hline
    26B &
    Credit Card/Automated Teller Machine Fraud &
    Property\\
    \hline
    26C &
    Extortion/Blackmail &
    Property\\
    \hline
    26D &
    Welfare Fraud &
    Property\\
    \hline
    26E &
    Wire Fraud &
    Property\\
    \hline

    Gambling Offenses & & \\
    \hline
    39A &
    Betting/Wagering &
    Property\\
    \hline
    39B &
    Operating/Promoting/Assisting Gambling &
    Property\\
    \hline
    39C &
    Gambling Equipment Violations &
    Property\\
    \hline
    39D &
    Sports Tampering &
    Property\\
    \hline

    Homicide Offenses & & \\
    \hline
    09A &
    Murder \& Non-negligent Manslaughter &
    Property\\
    \hline
    09B &
    Negligent Manslaughter &
    Property\\
    \hline
    09C &
    Justifiable Homicide &
    Person/Not a Crime\\
    \hline

    Kidnapping/Abduction & & \\
    \hline
    100 &
    Kidnapping/Abduction &
    Person\\
    \hline

    Larceny/Theft Offenses & & \\
    \hline
    23A &
    Pocket-picking &
    Person\\
    \hline
    23B &
    Purse-snatching &
    Person\\
    \hline
    23C &
    Shoplifting &
    Person\\
    \hline
    23D &
    Theft From Building &
    Person\\
    \hline
    23E &
    Theft From Coin-Operated Machine or Device &
    Person\\
    \hline
    23F &
    Theft From Motor Vehicle &
    Person\\
    \hline
    23G &
    Theft of Motor Vehicle Parts or Accessories &
    Person\\
    \hline
    23H &
    All Other Larceny &
    Person\\
    \hline

    Motor Vehicle & Theft & \\
    \hline
    240 &
    Motor Vehicle Theft &
    Property\\
    \hline

    Pornography / & Obscene Material & \\
    \hline
    370 &
    Pornography/Obscene Material &
    Society\\
    \hline

    Prostitution Offenses & & \\
    \hline
    40A &
    Prostitution &
    Society\\
    \hline
    40B &
    Assisting or Promoting Prostitution &
    Society\\
    \hline

    Robbery & & \\
    \hline
    120 &
    Robbery &
    Property\\
    \hline

    Sex Offenses, Forcible & & \\
    \hline
    11A &
    Forcible Rape &
    Person\\
    \hline
    11B &
    Forcible Sodomy &
    Person\\
    \hline
    11C &
    Sexual Assault With An Object &
    Person\\
    \hline
    11D &
    Forcible Fondling &
    Person\\
    \hline

    Sex Offenses, Nonforcible & & \\
    \hline
    36A &
    Incest &
    Person\\
    \hline
    36B &
    Statutory Rape &
    Person\\
    \hline

    Stolen Property Offenses & & \\
    \hline
    280 &
    Stolen Property Offenses &
    Property\\
    \hline

    Weapon Law Violations & & \\
    \hline
    520 &
    Weapon Law Violations &
    Society\\
    \hline
\caption{FBI supporting dataset attributes group A}
\end{longtable}
\end{center}

Group B Offenses
\begin{center}
\begin{longtable}{|m{8em} m{16em} m{7em}|}
    \hline
    Offense Code & 
    Offense Description & 
    Crime Against\\
    \hline
    90A & 
    Bad Checks & 
    Property\\
    \hline
    90B & 
    Curfew/Loitering/Vagrancy Violations & 
    Society\\
    \hline
    90C & 
    Disorderly Conduct & 
    Society\\
    \hline
    90D & 
    Driving Under the Influence & 
    Society\\
    \hline
    90E & 
    Drunkenness & 
    Society\\
    \hline
    90F & 
    Family Offenses, Nonviolent & 
    Society\\
    \hline
    90G & 
    Liquor Law Violations & 
    Society\\
    \hline
    90H & 
    Peeping Tom & 
    Society\\
    \hline
    90I & 
    Runaway & 
    Not a Crime\\
    \hline
    90J & 
    Trespass of Real Property & 
    Society\\
    \hline
    90Z & 
    All Other Offenses & 
    Person, Property, or Society\\
    \hline
\caption{FBI supporting dataset attributes group A}
\end{longtable}
\end{center}



TO DO:\newline
1- data import \newline
import data to R, break data to date \& time, remove X and y coordinate, remove location\newline
DONE!!!!!!! \newline
2- remove empty data and remove outliers \newline
3- see how data is shaped, what correlations are formed, etc... \newline